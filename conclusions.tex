In this study, four different solutions to the problem of VR head collisions were described and evaluated. The screen fade, delayed push-back, instant push-back, and teleportation methods were implemented and then tested in a specially designed virtual environment. 20 participants were recruited and completed the experiment, in which they had to collide with 10 various objects while using one of the solution to the problem. After testing each method, they filled in post-test questionnaires that measured what effects these solutions have on many different factors. The questionnaires results were analyzed using statistical tests and interpreted to determine which solution should be used in future VR applications. All objectives of the study were fulfilled and the results confirmed that some solutions to the problem of VR head collisions have more positive effect than others on VR sickness, the sense of presence, and the usability level.

Overall, the screen fade method turned out to be the most efficient one of the four solutions and should be the first choice for VR developers. It is characterized by a simple implementation that can be easily integrated into any VR application. It has more positive effect than the delayed push-back method in terms of VR sickness, sense of presence, tiredness, enjoyment, frustration, and general usability. It also achieves better scores than the teleportation method in terms of usability, enjoyment, and sense of presence. The instant push-back method had mostly neutral results. It only has more positive effect than the delayed push-back method in terms of usability, feelings of control, and sense of presence. No significant differences were found between the delayed push-back and the screen fade method. It is recommended to use this solution in place of the screen fade method if the developer does not want to obstruct the vision of the players at any time. The teleportation method also had neutral scores for VR sickness. However, it achieved substandard scores for the sense of presence and usability, and it generally should not be the preferred solution. The delayed push-back method ranked the worst in all categories and should be avoided by VR developers.

Although the study has successfully demonstrated that some solutions have more positive effects than others on many different factors, the research was limited in several ways. The obtained results are based on specific, most commonly used implementation of the methods. Slight modifications to these solutions may completely change how they are perceived by the users. Furthermore, due to time constraints, only four different solutions to the problem of VR head collisions were examined. There are many more techniques that can be implemented and compared with the solutions described in this study. Another limitation is that the number of participants that completed the experiment was relatively small, and the duration of tests for each method lasted only a couple of minutes. The symptoms of VR sickness tend to increase the longer the VR experience lasts, which can affect the experiment results. In the future, it is advised to conduct similar experiments with a larger pool of participants and with longer duration of tests if VR sickness is the main focus.

The screen fade method turned out to be the best in terms of many factors, however, this solution still has one minor issue that sometimes occurs in teleportation locomotion techniques. The users can unintentionally teleport to some head-object colliding position, and then be stuck with obscured screen without knowing in what direction they should move to escape the darkness. Further research in this field might explore the ways how to prevent this situation. For example, some visual cues that guide the users might help them move out of the collisions. In the case of other promising solutions to the problem of VR head collisions, a future study investigating the object fade method is strongly recommended. Object fade is a technique in which, rather than fading the whole screen to black, only parts of the colliding objects fade out as the camera gets closer to them. Due to the similarity to the screen fade method and the possibility to add some interesting fading visual effects, this solution seems now the most promising.