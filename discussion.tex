\section{VR Sickness}

The first research question in this study sought to determine how the proposed solutions to the problem of VR head collisions affect virtual reality sickness. The results support the hypothesis that some solutions have more positive effect than others. Both disorientation and total SSQ scores were significantly lower for the screen fade method than the delayed push-back method. No differences were found between any other of the four methods. There were also no significant differences revealed in the case of SSQ nausea and oculumotor scores. However, the SSQ disorientation scores were much higher than the oculomotor and nausea scores, which further support the idea that VR sickness tends to be characterized by the disorientation disturbances. 

The high scores of the delayed push-back method explain the complaints that can be found in many internet forums. This method was chosen by some VR developers and the players reported that it had great impact on virtual reality sickness. The most likely reason for the high scores is that this method for a brief moment takes control of camera movements away from the player. It is proven that any unexpected accelerations and movements of the camera not initiated by the player's physical movement are some of the main causes of VR sickness. In this study, the method was implemented in the way that allowed the player to move his head during the duration of the push-back effect as opposed to completely disabling the head tracking. The SSQ scores would most likely be even higher if the method was implemented in a different way.

In contrast to earlier findings, no evidence of difference between the screen fade method and the instant push-back was detected. The previous research found that the instant push-back achieved higher SSQ scores than the screen fade method \cite{COMPARISONCOLLISION}. Possible explanations for these results may be the longer duration of the experiment, different experiment design, or slightly different implementations of the methods. The symptoms of VR sickness tend to increase the longer the VR experience lasts. In the previous research, the participants tested different solutions for 10 minutes by playing a simple game in virtual environment with narrow corridors.

\section{Sense of Presence}

The second research question sought to find how the studied methods affect the sense of presence. The results again support the hypothesis that some methods have more positive effect than others. The screen fade method leads to a significantly higher sense of presence than the teleportation and delayed push-back methods. Moreover, the instant push-back method provides a significantly higher sense of presence than the delayed push-back method. However, no significant difference was found between the screen fade method and the instant push-back method.

There are several possible explanations for these results. During testing the screen fade method, the participants often commented that fading the screen to black feels the most natural to them. Some participants explained that this effect resembles the darkness due to closing the eyes when their heads collide with objects in the real world. During testing the instant push-back method, some participants commented that they felt like they are pushing themselves away from the object by using their heads, which again in some way resembles how their feet would be pushed backwards on a slippery floor in the real world. Moreover, the screen fade and instant push-back methods are the only techniques where the players are never able to see the insides of objects and unexpected clipping artifacts. In the delayed push-back and teleportation methods, the players are able to see the insides of objects for a brief moment before the methods start to work and relocate them to a new position.

\section{Usability}

The third and final research question in this study sought to identify how usable are the proposed solutions. The results back the hypothesis that some solutions have higher level of usability than others. The exact same situation as in the sense of presence case repeated for total usability scores: the screen fade method leads to a significantly higher level of usability than the teleportation and delayed push-back methods, and the instant push-back method provides a significantly higher level of usability than the delayed push-back method. Once again, no significant difference was found between the screen fade method and the instant push-back method.

The participants of the experiment were not told how the methods work and had to discover it on their own. However, they had no trouble in operating the methods and quickly understood how they function. No significant differences were found between the methods in this case. This result may be explained by the fact that each method is triggered in a similar way. Some of the participants were slightly confused at first when they tested the teleportation and delayed push-back methods. In these cases, the methods do not work instantly after colliding with the headset, but after a second of constant collision. However, after a brief investigation the participants were able to trigger the effect and had no trouble finishing the experiment with the remaining objects.

Although there were no significant differences found in the required effort scores, the scores for tiredness were significantly influenced by the method of handling collisions. This rather contradictory result may be due to the effects of VR sickness. The results for tiredness mirror these found in VR sickness section: the screen fade method achieved significantly lower scores than the delayed push-back method. While the participants felt like the methods are equally demanding, the higher tiredness scores may be attributed to the increases in VR sickness symptoms.

The results show that the participants felt more in control while they tested the instant push-back method compared to the delayed push-back method. This result may be explained by the fact that the users lose some control of the movement for a longer duration in the delayed push-back method. In the instant push-back method, the users manually initiate the effect by pushing their heads forward in the direction of the collision and can stop doing it at any time. The vision is also never obscured for a brief moment like in the teleportation and screen fade methods and the users can see what is happening on the screen all the time.

Another important finding is that the screen fade method is more enjoyable than the delayed push-back and teleportation methods. Furthermore, the results indicate that the delayed push-back method is more frustrating to experience than the screen fade method. There are a couple of possible explanations for these results. During testing the screen fade method, some participants liked the visual effect of gradual fading. They repeated moving the head in and out of the object, even if its color was already changed to green. Although the participants mostly enjoyed discovering how the teleportation and delayed push-back methods work, after experiencing them for the first time, some participants found it frustrating to wait a full second to trigger the effect with the remaining objects.