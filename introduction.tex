\section{Background}

Virtual reality (VR) has become increasingly popular over the past few years. Several tech companies, such as Google, Facebook, Sony, Samsung, or HTC, released many affordable consumer-grade VR systems. Room-scale and seated VR experiences are now easily accessible to anyone interested in this trending technology. Besides entertainment purposes, today's VR has found applications in education, training, healthcare, architecture, and many other areas. However, while VR technology has seen rapid development and growth in recent years, it still has much to improve upon. Modern VR systems lack the ability to simulate realistic haptic sensations and the sense of smell. They also do not have eye tracking capability, which would provide a whole new way to interact with VR content. The current challenge is to increase the sense of presence in virtual environments using the available technology. Another concern is virtual reality sickness that can occur during long exposures to virtual environments. This sickness is mostly characterized by disorientation disturbances and its symptoms resemble the ones seen in motion sickness.

There has been a lot of study into reducing the effects of virtual reality sickness. Because one of the main reasons for the sickness is a mismatch between the virtual and real motions, most of the research has been focused on various locomotion techniques. These techniques have been thoroughly studied in terms of VR sickness, sense of presence, usability, and user experience. A selected handful of these techniques have been chosen as the most comfortable for the users and are now implemented in almost all VR applications. However, there is still much room for improvement, and there are still many issues to solve. This study focuses on one particular problem related to head tracking systems.

The head tracking systems play a big part in VR movement. They are used to estimate the position and rotation of VR headset. The users can now control the virtual camera with their physical head movements. There is no longer a need for a mouse or another input device to look around in the virtual world. Head tracking greatly increases the immersion experienced by the users and can be successfully applied to many locomotion techniques. Unfortunately, it also introduces a significant problem with head collisions that is not normally present in video games. The problem occurs when the user's virtual head should collide with some virtual object, but the user keeps moving his head forward in the real world without any obstructions. The physical collision with the object is currently impossible to simulate due to the lack of advanced haptic systems. Instead, the user sees the insides of the virtual object and unexpected clipping artifacts, which can break the gameplay and immersion of the game. There is a need for solution to the problem that would prevent this situation from happening in future VR applications.

\section{Research Objectives}

Because head tracking is a relatively new technology and most recent studies on VR movement were focused on comparisons of various locomotion techniques, there is a general lack of research into the described problem of VR head collisions. In recent years, VR developers were forced to develop their own solution to the problem if they wanted to prevent the players from clipping through walls and other objects. This led to the creation of many techniques, each with their advantages and disadvantages, and each with their supporters and critics. The main goal of this study is to examine the most promising solutions and to decide which one of them is best suited for future VR games. However, the quality of VR experience can be measured by many factors and some developers may consider different aspects to be more important than others. Most researchers that study VR movement take into consideration VR sickness, usability of examined methods, and the sense of presence in virtual environments. These factors are often measured in experiments using popular and proven questionnaires. 

To summarize, the objective of this study is to examine different solutions to the VR head collisions problem and to determine what effects they have on VR sickness, the sense of presence, and the usability level. The results received from questionnaires measuring these three factors should help developers decide which one of the examined solutions they want use in their VR applications.

\section{Scope of the Study}

Due to time constraints, only four most commonly used solutions to the problem of VR head collisions are examined in this study: screen fade, delayed push-back, instant push-back, and teleportation. In the screen fade method, the whole screen fades to black when the head collision is detected. In the delayed push-back method, the user can look inside some object for a brief moment until he is slowly pushed backwards if the collision is maintained. In the instant push-back method, instead of the slow push-back, the user is instantly moved backwards by a collision vector. In the teleportation method, once again the user can look inside the object for a brief moment before he is teleported to a nearby collision-free location. These techniques are described in details in the literature review and methodology chapters. There are no official names for these solutions and the ones used in this study were chosen based on their description. Therefore, in future research studies these specific techniques can occur under different names.

The next chapter is literature review. It presents an in depth investigation of the literature relevant to the methods for implementing user movement in VR. It outlines the current state of art of virtual reality technologies and examines various locomotion techniques with different input methods. It also describes in details the problem of VR sickness and the problem of VR head collisions, which is the main focus of this study. The methodology chapter presents the research questions that this study seeks to answers and defines the hypotheses that it tries to prove. It also contains the description of questionnaires, implementation, design, participants, and equipment used in the prepared experiment. The methodology section is followed by the results obtained from the experiment, which were analyzed using various statistical methods, and the discussion chapter that interprets the data. Finally, the conclusions chapter summarizes the major findings, highlights the limitations of this study, and ends with recommendations for future research.