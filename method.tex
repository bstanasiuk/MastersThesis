\section{Research Questions and Hypotheses}

There is a general lack of research into the VR head collisions problem. Current VR developers experiment with various solutions in their games, and all of these methods have their advantages and disadvantages. There is a disagreement in VR developers community over which solution should be used in future VR games. Some developers wonder if it is even worth the effort to implement any solution at all. Therefore, the goal of this study is to examine different solutions and to determine which one of them is best suited for modern VR games.
 
There are several possible solutions to the VR head collisions problem, and they can be implemented in many different ways. Due to time constraints, only the four most popular methods were chosen for this study. These are:

\begin{itemize}
\item Screen fade: When the head collision is detected, the whole screen fades to black in the span of a second.
\item Object fade: As the user's head gets closer to the collision, parts of the surrounding objects start to fade out.
\item Camera collider: The player's head is surrounded by a rigidbody collider. When the head collision is detected, overlapping with the object is prevented.
\item Camera push-back: If the player keeps colliding with the object for longer than a second, he is slowly pushed backwards until he leaves the object's boundaries.
\end{itemize}

The quality of VR experience is affected by many factors. VR sickness, the sense of presence in virtual world, and the usability of the user interface are some of the main considerations in designing enjoyable VR experience. Some developers consider different factors to be more important than others. For this reason, the proposed solutions to the VR head collisions problem are examined from three different perspectives. The answers to the following research questions will help VR developers decide which particular solution to use:

\begin{itemize}
\item RQ1: How the proposed solutions to the VR head collisions problem affect the virtual reality sickness?
\item RQ2: How the proposed solutions to the VR head collisions problem affect the sense of presence?
\item RQ3: How usable are the proposed solutions to the VR head collisions problem?
\end{itemize}

The three null hypotheses corresponding to the research questions are as follows:

\begin{itemize}
\item H$_{\text{01}}$: All proposed solutions to the VR head collisions problem have the same effect on the virtual reality sickness.
\item H$_{\text{02}}$: All proposed solutions to the VR head collisions problem have the same effect on the sense of presence.
\item H$_{\text{03}}$: All proposed solutions to the VR head collisions problem have the same level of usability.
\end{itemize}

The null hypotheses are tested against the following three alternative hypotheses:

\begin{itemize}
\item H$_{\text{a1}}$: Some proposed solutions to the VR head collisions problem have more positive effect than others on the virtual reality sickness.
\item H$_{\text{a2}}$: Some solutions to the VR head collisions problem have more positive effect than others on the sense of presence.
\item H$_{\text{a3}}$: Some solutions to the VR head collisions problem have higher level of usability than others.
\end{itemize}

\section{Questionnaires}

\subsection{Demographic Questionnaire}

\subsection{Simulator Sickness Questionnaire}

\subsection{Presence Questionnaire}

\subsection{Usability Questionnaire}

\section{Implementation}

\subsection{Virtual Environment}

\subsection{Screen Fade}

\subsection{Object Fade}

\subsection{Camera Collider}

\subsection{Camera Push-Back}

\section{Experiment}

The study employed a within-subject design. The independent variable was the solution to the VR head collisions problem, and it had four levels: screen fade, object fade, camera collider, camera push-back. Due to time constraints, the participants were required to complete the whole experiment with every method in one sitting. The order of tested solutions was assigned randomly with Latin Square counterbalancing. 

\subsection{Equipment}

\subsection{Participants}

\subsection{Procedure}